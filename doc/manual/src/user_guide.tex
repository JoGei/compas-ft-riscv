

\section{Usage Guide}

Various \ac{sihft} passes can be invoked via command line arguments to the LLVM static compiler \texttt{llc} program.
This chapter assumes that the LLVM compiler has been installed as per instructions in earlier sections.
We refer the installation location of this compiler to \texttt{<llvm\_install>} in rest of the chapter.

\subsection{\ac{sihft} Compilation}
Each source file (translation unit) in C/C++ project has to be first compiled to LLVM IR code using \texttt{clang}.
Following command can be used to generate human readable LLVM IR code representation of given C source file.
\begin{framed}
 \begin{lstlisting}[language=bash, basicstyle=\small\ttfamily]
$ <llvm_install>/bin/clang
    -emit-llvm -S <other options>
    main.c -o main.ll
  \end{lstlisting}
\end{framed}

The next step in the compilation is to pass the LLVM IR file to \texttt{llc} program:\\
\begin{lstlisting}[language=bash, basicstyle=\small\ttfamily, frame=single]
$ <llvm_install>/bin/llc
    <SIHFT options> <other options>
    main.ll -o main_hardened.s
\end{lstlisting}

\texttt{<SIHFT options>} refer to various \ac{sihft} transformations that we have implemented. A brief overview of these
options is provided in Table~\ref{tab:sihft-options}.

\begin{table}[htb]
 \centering
 \caption{Supported \ac{sihft} options}
 \label{tab:sihft-options}

 \begin{tabular}{|l|l|}
  \hline
  \textbf{Option} & \textbf{Description}                                     \\
  \hline
  -NZDC=foo,bar   & apply NZDC transformation on \textit{foo,bar} functions  \\
  -SWIFT=foo,bar  & apply SWIFT transformation on \textit{foo,bar} functions \\
  -RASM=foo,bar   & apply RASM transformation on \textit{foo,bar} functions  \\
  -CFCSS=foo,bar  & apply CFCSS transformation on \textit{foo,bar} functions \\
  -FGS            & use fine-grain scheduling for NZDC code                  \\
  -REPAIR         & use REPAIR transformation on NZDC code                   \\
  \hline
 \end{tabular}
\end{table}

The final step in code generation is to use \texttt{gcc} for RISCV architectures to assemble and link the
assembly code for a particular RISCV processor.

\subsubsection{Example}
For the sake of example, assume we want to protect the \texttt{crc} program from the MiBench suite. The
compiled program is to run on the RISCV spike simulator. The C project contains the following C files:

\begin{itemize}
 \item{crc.h: header for CRC library functions}
 \item{crc.c: source for CRC library functions}
 \item{main.c: contains the \texttt{main} program to invoke crc library functions}
\end{itemize}

Following commands are issued on bash shell to create the RISCV binary with
\begin{itemize}
 \item NZDC, RASM protection on \texttt{crcSlow} function
 \item RASM protection on \texttt{crcFast} function
 \item NZDC, CFCSS protection on \texttt{main} function
\end{itemize}

\begin{framed}
 \begin{lstlisting}[language=bash, basicstyle=\small\ttfamily]
$ cd <crc-project-source>

$ <clang> -emit-llvm -O2 -S --target=riscv64
    -march=rv64gc -mabi=lp64d
    -isystem <riscv64-gcc>/riscv64-unknown-elf/
    main.c -o main.ll
$ <clang> -emit-llvm -O2 -S --target=riscv64
    -march=rv64gc -mabi=lp64d
    -isystem <riscv64-gcc>/riscv64-unknown-elf/
    crc.c -o crc.ll

$ <llc> -O2 -march=riscv64 -mattr=+m,+a,+d,+c
    -NZDC=main -CFCSS=main
    main.ll -o main_hardened.s
$ <llc> -O2 -march=riscv64 -mattr=+m,+a,+d,+c
    -NZDC=crcSlow -RASM=crcSlow,crcFast
    crc.ll -o crc_hardened.s

$ <riscv64-gcc>/bin/riscv64-unknown-elf-gcc -O2
    -march=rv64gc -mabi=lp64d
    main_hardened.s crc_hardened.s -o crc_hardened.elf

$ spike pk crc_hardened.elf

\end{lstlisting}
\end{framed}

% \newpage
% \section{Configuration via TOML script}
